\documentclass[a4paper]{article}

%% Language and font encodings
\usepackage[brazil]{babel}
\usepackage[utf8x]{inputenc}
\usepackage[T1]{fontenc}

%% Sets page size and margins
\usepackage[a4paper,top=2cm,bottom=2cm,left=3cm,right=3cm,marginparwidth=1.75cm]{geometry}

%% Useful packages
\usepackage{amsmath}
\usepackage{graphicx}
\usepackage[colorinlistoftodos]{todonotes}
\usepackage[colorlinks=true, allcolors=blue]{hyperref}

\title{IF673 - Lógica para Computação}
\author{Vinícius Dantas Januário}

\begin{document}
\maketitle

\section{Introdução}

Lógica é uma cadeira do segundo período do curso de Ciência da computação da UFPE e está inserida na grande área dos fundamentos matemáticos da computação. A disciplina estuda o raciocínio dedutivo no âmbito da lógica matemática, que testa a validade de argumentos com ferramentas da matemática. Os seguintes tópicos são abordados no decorrer da disciplina: 

\begin{itemize}
\item Potencialidades e limites do método formal-dedutivo de representação e raciocínio sobre uma "realidade".
\item A fundamentação das noções de prova e refutação da validade de argumentos.
\item Os fundamentos da representação simbólica, e da noção de consequência lógica.
\end{itemize}

\section{Relevância}

O desenvolvimento dos conceitos fundamentais da Ciência da Computação são de consequência direta da lógica matemática. Como exemplo, o raciocínio dedutivo num sistema formal, criado pelo matemático Gottlob Frege, está relacionado com os fundamentos lógicos de um computador abstrato, conhecido como a Maquina de Turing, concebido pelo matemático britânico de mesmo nome, Alan Turing. As linguagens de programação (importante para um cientista da computação) tem como base a representação simbólica da lógica matemática.

\section{Relação com outras disciplinas}

A seguinte tabela mostra uma relação entre a cadeira de lógica e outras cadeiras do curso, sendo a primeira pré-requisito para lógica e as outras duas tem lógica como pré-requisito.

\begin{table}[h]
\centering
\caption{Relação de Lógica para Computação com outras cadeiras}
\label{my-label}
\begin{tabular}{|l|l|}
\hline
IF670 - Matemática Discreta para Computação & \begin{tabular}[c]{@{}l@{}}Sendo pré-requisito para a cadeira \\ de lógica, a cadeira de matemática \\ discreta, como o próprio nome sugere, \\ estuda as grandezas matemáticas \\ discretas e finitas, como os números \\ inteiros, recursividade, grafos e árvores.\end{tabular}                              \\ \hline
IF689 - Informática Teórica                 & \begin{tabular}[c]{@{}l@{}}Pode ser considerada uma continuação\\  da cadeira de lógica, já que esta cadeira \\ aborda os principais conceitos da \\ informática. Como explicado \\ anteriormente, esses conceitos \\ fundamentais da computação foram \\ derivados a partir da lógica matemática.\end{tabular} \\ \hline
IF682 - Engenharia de Software e Sistemas   & \begin{tabular}[c]{@{}l@{}}É uma disciplina que envolve a aplicação\\ dos conceitos até então estudados para a \\ construção de softwares e sistemas, que \\ requer o entendimento de linguagens de\\ programação, que se relaciona \\ diretamente com a lógica.\end{tabular}                                   \\ \hline
\end{tabular}
\end{table}

\clearpage

\section{Referências}

Como bibliografia oficial, a cadeira utiliza os seguintes livros: 

\begin{enumerate}
\item Logic and Structure\cite{van2012logic}
\item A Shorter Model Theory\cite{hodges1997shorter}
\item Language, proof and logic\cite{barwise1999language}
\end{enumerate}

\bibliographystyle{alpha}
\bibliography{vdj.bib}

\end{document}